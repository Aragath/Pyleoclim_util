%% Generated by Sphinx.
\def\sphinxdocclass{report}
\documentclass[letterpaper,10pt,english]{sphinxmanual}
\ifdefined\pdfpxdimen
   \let\sphinxpxdimen\pdfpxdimen\else\newdimen\sphinxpxdimen
\fi \sphinxpxdimen=49336sp\relax

\usepackage[margin=1in,marginparwidth=0.5in]{geometry}
\usepackage[utf8]{inputenc}
\ifdefined\DeclareUnicodeCharacter
  \DeclareUnicodeCharacter{00A0}{\nobreakspace}
\fi
\usepackage{cmap}
\usepackage[T1]{fontenc}
\usepackage{amsmath,amssymb,amstext}
\usepackage{babel}
\usepackage{times}
\usepackage[Bjarne]{fncychap}
\usepackage{longtable}
\usepackage{sphinx}

\usepackage{multirow}
\usepackage{eqparbox}

% Include hyperref last.
\usepackage{hyperref}
% Fix anchor placement for figures with captions.
\usepackage{hypcap}% it must be loaded after hyperref.
% Set up styles of URL: it should be placed after hyperref.
\urlstyle{same}

\addto\captionsenglish{\renewcommand{\figurename}{Fig.\@ }}
\addto\captionsenglish{\renewcommand{\tablename}{Table }}
\addto\captionsenglish{\renewcommand{\literalblockname}{Listing }}

\addto\extrasenglish{\def\pageautorefname{page}}

\setcounter{tocdepth}{2}



\title{Pyleoclim Documentation}
\date{Jul 12, 2017}
\release{0.3.0}
\author{Deborah Khider, Julien Emile-Geay, Feng Zhu}
\newcommand{\sphinxlogo}{}
\renewcommand{\releasename}{Release}
\makeindex

\begin{document}

\maketitle
\sphinxtableofcontents
\phantomsection\label{\detokenize{index::doc}}


Contents:


\chapter{Pyleoclim}
\label{\detokenize{Introduction:welcome-to-pyleoclim-s-documentation}}\label{\detokenize{Introduction::doc}}\label{\detokenize{Introduction:pyleoclim}}

\section{What is it?}
\label{\detokenize{Introduction:what-is-it}}
Pyleoclim is a Python package primarily geared towards the analysis and visualization of paleoclimate data.
Such data often come in the form of timeseries with missing values and age uncertainties, and the package
includes several low-level methods to deal with these issues, as well as high-level methods that re-use those
to perform scientific workflows.

The package assumes that the data are stored in the Linked Paleo Data (\href{http://www.clim-past.net/12/1093/2016/}{LiPD})
format and makes extensive use of the \href{http://nickmckay.github.io/LiPD-utilities/}{LiPD utilities}. The package
is aware of age ensembles stored via LiPD and uses them for time-uncertain analyses very much like \href{http://nickmckay.github.io/GeoChronR/}{GeoChronR}.

\sphinxstylestrong{Current Capabilities:}
\begin{itemize}
\item {} 
binning

\item {} 
interpolation

\item {} 
plotting maps, timeseries, and basic age model information

\item {} 
paleo-aware correlation analysis (isopersistent, isospectral, and classical t-test)

\item {} 
weighted wavelet Z transform (WWZ)

\end{itemize}

\sphinxstylestrong{Future capabilities:}
\begin{itemize}
\item {} 
paleo-aware singular spectrum analysis (AR(1) null eigenvalue identification, missing data)

\item {} 
spectral analysis (Multi-Taper Method, Lomb-Scargle)

\item {} 
cross-wavelet analysis

\item {} 
index reconstruction

\item {} 
climate reconstruction

\item {} 
ensemble methods for most of the above

\end{itemize}


\section{Version Information}
\label{\detokenize{Introduction:version-information}}
\begin{DUlineblock}{0em}
\item[] 0.3.0: Compatibility with LiPD 1.3 and Spectral module added
\item[] 0.2.5: Fix error on loading (Looking for Spectral Module)
\item[] 0.2.4: Fix load error from init
\item[] 0.2.3: Freeze LiPD version to 1.2 to avoid conflicts with 1.3
\item[] 0.2.2: Change progressbar to tqdm and add standardization function
\item[] 0.2.1: Update package requirements
\item[] 0.2.0: Restructure the package so that the main functions can be called without the use of a LiPD files and associated timeseries objects.
\item[] 0.1.4: Rename functions using camel case convention and consistency with LiPD utilities version 0.1.8.5
\item[] 0.1.3: Compatible with LiPD utilities version 0.1.8.5
\item[]
\begin{DUlineblock}{\DUlineblockindent}
\item[] Function openLiPD() renamed openLiPDs()
\end{DUlineblock}
\item[] 0.1.2: Compatible with LiPD utilities version 0.1.8.3
\item[]
\begin{DUlineblock}{\DUlineblockindent}
\item[] Uses Basemap instead of cartopy
\end{DUlineblock}
\item[] 0.1.1: Freezes the package prior to version 0.1.8.2 of LiPD utilities
\item[] 0.1.0: First release
\end{DUlineblock}


\section{Installation}
\label{\detokenize{Introduction:installation}}
Python v3.4+ is required. Tested with Python v3.5
Pyleoclim is published through Pypi and easily installed via pip:

\begin{sphinxVerbatim}[commandchars=\\\{\}]
\PYG{n}{pip} \PYG{n}{install} \PYG{n}{pyleoclim}
\end{sphinxVerbatim}


\section{Quickstart guide}
\label{\detokenize{Introduction:quickstart-guide}}\begin{enumerate}
\item {} 
Open your command line application (Terminal or Command Prompt)

\item {} 
Install with command:

\end{enumerate}
\begin{quote}

pip install pyleoclim
\end{quote}
\begin{enumerate}
\setcounter{enumi}{2}
\item {} 
Wait for installation to complete, then:

\end{enumerate}
\begin{enumerate}
\item {} 
Import the package into your favorite Python environment (we recommend the use of Spyder, which comes standard with the Anaconda build)

\item {} 
Use Jupyter Notebook to go through the tutorial contained in the \href{https://github.com/LinkedEarth/Pyleoclim\_util/tree/master/Example}{PyleolimQuickstart.ipynb}

\end{enumerate}


\section{Requirements}
\label{\detokenize{Introduction:requirements}}\begin{itemize}
\item {} 
LiPD v0.2.2+

\item {} 
pandas v0.19+

\item {} 
numpy v1.12+

\item {} 
matplotlib v2.0+

\item {} 
Basemap v1.0.7+

\item {} 
scipy v0.19.0+

\item {} 
statsmodel v0.8.0+

\item {} 
seaborn v0.7.0+

\item {} 
scikit-learn v0.17.1+

\item {} 
tqdm v4.14.0+

\item {} 
pathos v0.2.0+

\end{itemize}

The installer will automatically check for the needed updates.


\section{Further information}
\label{\detokenize{Introduction:further-information}}
\begin{DUlineblock}{0em}
\item[] GitHub: \url{https://github.com/LinkedEarth/Pyleoclim\_util}
\item[] LinkedEarth: \url{http://linked.earth}
\item[] Python and Anaconda: \url{http://conda.pydata.org/docs/test-drive.html}
\item[] Jupyter Notebook: \url{http://jupyter.org/}
\end{DUlineblock}


\section{Contact}
\label{\detokenize{Introduction:contact}}
Please report issues to \href{mailto:linkedearth@gmail.com}{linkedearth@gmail.com}


\section{License}
\label{\detokenize{Introduction:license}}
The project is licensed under the \href{https://github.com/LinkedEarth/Pyleoclim\_util/blob/master/license}{GNU Public License} .


\section{Disclaimer}
\label{\detokenize{Introduction:disclaimer}}
This material is based upon work supported by the U.S. National Science Foundation under Grant Number
ICER-1541029. Any opinions, findings, and conclusions or recommendations expressed in this material are those
of the investigators and do not necessarily reflect the views of the National Science Foundation.


\chapter{Main Functions}
\label{\detokenize{Main:main-functions}}\label{\detokenize{Main::doc}}

\section{Using Pyleoclim with a LiPD file}
\label{\detokenize{Main:using-pyleoclim-with-a-lipd-file}}

\subsection{Getting started}
\label{\detokenize{Main:getting-started}}
Pyleoclim relies heavily on the concept of timeseries objects introduced in
\href{http://www.clim-past.net/12/1093/2016/}{LiPD} and implemented in the
\href{http://nickmckay.github.io/LiPD-utilities/}{LiPD utilities}.

Briefly, timeseries objects are dictionaries containing the ChronData values and
PaleoData values as well as the metadata associated with the record. If one record
has three ProxyObservations (e.g., Mg/Ca, d18O, d13C) then it will have three timeseries
objects, one for each of the observations.

The LiPD utilities function lipd.extractTs() returns a list of dictionaries for
the selected LiPD files, which need to be passed to Pyleoclim along with the path
to the directory containing the LiPD files.
\begin{description}
\item[{This is done through the functions pyleoclim.readLiPD() and pyleoclim.extracTs(),}] \leavevmode
which are lightweight versions of their counterparts in the LiPD utilities:

\end{description}
\index{readLipd() (in module pyleoclim)}

\begin{fulllineitems}
\phantomsection\label{\detokenize{Main:pyleoclim.readLipd}}\pysiglinewithargsret{\sphinxcode{pyleoclim.}\sphinxbfcode{readLipd}}{\emph{usr\_path='`}}{}
Read Lipd files into a dictionary

This function is based on the function of the same name in the LiPD utilities.
Sets the dictionary as global variable so that it doesn't have to be provided
as an argument for every function.
\begin{quote}\begin{description}
\item[{Parameters}] \leavevmode
\sphinxstyleliteralstrong{usr\_path} (\sphinxstyleliteralemphasis{str}) -- The path to a directory or a single file. (Optional argument)

\item[{Returns}] \leavevmode
lipd\_dict - a dictionary containing the LiPD library

\end{description}\end{quote}

\end{fulllineitems}

\index{extractTs() (in module pyleoclim)}

\begin{fulllineitems}
\phantomsection\label{\detokenize{Main:pyleoclim.extractTs}}\pysiglinewithargsret{\sphinxcode{pyleoclim.}\sphinxbfcode{extractTs}}{\emph{lipds=None}}{}
Extract timeseries dictionary

This function is based on the function of the same name in the LiPD utilities.
Set the dictionary as a global variable so that it doesn't have to be
provided as an argument for every function.
\begin{quote}\begin{description}
\item[{Parameters}] \leavevmode\begin{itemize}
\item {} 
\sphinxstyleliteralstrong{lipds} (\sphinxstyleliteralemphasis{dict}) -- A dictionary of LiPD files obtained through the

\item {} 
\sphinxstyleliteralstrong{function} ({\hyperref[\detokenize{Main:pyleoclim.readLipd}]{\sphinxcrossref{\sphinxstyleliteralemphasis{readLipd}}}}) -- 

\end{itemize}

\item[{Returns}] \leavevmode
ts\_list - A list of timeseries object

\end{description}\end{quote}

\end{fulllineitems}



\subsection{Mapping}
\label{\detokenize{Main:mapping}}\index{mapAllArchive() (in module pyleoclim)}

\begin{fulllineitems}
\phantomsection\label{\detokenize{Main:pyleoclim.mapAllArchive}}\pysiglinewithargsret{\sphinxcode{pyleoclim.}\sphinxbfcode{mapAllArchive}}{\emph{lipd\_dict='`}, \emph{markersize=50}, \emph{background='shadedrelief'}, \emph{saveFig=False}, \emph{dir='`}, \emph{format='eps'}}{}
Map all the available records loaded into the workspace by archiveType.
\begin{description}
\item[{Map of all the records into the workspace by archiveType.}] \leavevmode
Uses the default color palette. Enter pyleoclim.plot\_default for detail.

\end{description}
\begin{quote}\begin{description}
\item[{Parameters}] \leavevmode\begin{itemize}
\item {} 
\sphinxstyleliteralstrong{lipd\_dict} (\sphinxstyleliteralemphasis{dict}) -- A dictionary of LiPD files. (Optional)

\item {} 
\sphinxstyleliteralstrong{markersize} (\sphinxstyleliteralemphasis{int}) -- The size of the markers. Default is 50

\item {} 
\sphinxstyleliteralstrong{background} (\sphinxstyleliteralemphasis{str}) -- Plots one of the following images on the map:
bluemarble, etopo, shadedrelief, or none (filled continents).
Default is shadedrelief.

\item {} 
\sphinxstyleliteralstrong{saveFig} (\sphinxstyleliteralemphasis{bool}) -- Default is to not save the figure

\item {} 
\sphinxstyleliteralstrong{dir} (\sphinxstyleliteralemphasis{str}) -- The absolute path of the directory in which to save the
figure. If not provided, creates a default folder called `figures'
in the LiPD working directory (lipd.path).

\item {} 
\sphinxstyleliteralstrong{format} (\sphinxstyleliteralemphasis{str}) -- One of the file extensions supported by the active
backend. Default is ``eps''. Most backend support png, pdf, ps, eps,
and svg.

\end{itemize}

\item[{Returns}] \leavevmode
The figure

\end{description}\end{quote}

\end{fulllineitems}

\index{mapLipd() (in module pyleoclim)}

\begin{fulllineitems}
\phantomsection\label{\detokenize{Main:pyleoclim.mapLipd}}\pysiglinewithargsret{\sphinxcode{pyleoclim.}\sphinxbfcode{mapLipd}}{\emph{timeseries='`}, \emph{countries=True}, \emph{counties=False}, \emph{rivers=False}, \emph{states=False}, \emph{background='shadedrelief'}, \emph{scale=0.5}, \emph{markersize=50}, \emph{marker='ro'}, \emph{saveFig=False}, \emph{dir='`}, \emph{format='eps'}}{}
Create a Map for a single record

Orthographic projection map of a single record.
\begin{quote}\begin{description}
\item[{Parameters}] \leavevmode\begin{itemize}
\item {} 
\sphinxstyleliteralstrong{timeseries} -- a LiPD timeseries object. Will prompt for one if not given

\item {} 
\sphinxstyleliteralstrong{countries} (\sphinxstyleliteralemphasis{bool}) -- Draws the country borders. Default is on (True).

\item {} 
\sphinxstyleliteralstrong{counties} (\sphinxstyleliteralemphasis{bool}) -- Draws the USA counties. Default is off (False).

\item {} 
\sphinxstyleliteralstrong{rivers} (\sphinxstyleliteralemphasis{bool}) -- Draws the rivers. Default is off (False).

\item {} 
\sphinxstyleliteralstrong{states} (\sphinxstyleliteralemphasis{bool}) -- Draws the American and Australian states borders.
Default is off (False)

\item {} 
\sphinxstyleliteralstrong{background} (\sphinxstyleliteralemphasis{str}) -- Plots one of the following images on the map:
bluemarble, etopo, shadedrelief, or none (filled continents).
Default is shadedrelief

\item {} 
\sphinxstyleliteralstrong{scale} (\sphinxstyleliteralemphasis{float}) -- useful to downgrade the original image resolution to
speed up the process. Default is 0.5.

\item {} 
\sphinxstyleliteralstrong{markersize} (\sphinxstyleliteralemphasis{int}) -- default is 50

\item {} 
\sphinxstyleliteralstrong{marker} (\sphinxstyleliteralemphasis{str}) -- a string (or list) containing the color and shape of the
marker. Default is by archiveType. Type pyleo.plot\_default to see
the default palette.

\item {} 
\sphinxstyleliteralstrong{saveFig} (\sphinxstyleliteralemphasis{bool}) -- default is to not save the figure

\item {} 
\sphinxstyleliteralstrong{dir} (\sphinxstyleliteralemphasis{str}) -- the full path of the directory in which to save the figure.
If not provided, creates a default folder called `figures' in the
LiPD working directory (lipd.path).

\item {} 
\sphinxstyleliteralstrong{format} (\sphinxstyleliteralemphasis{str}) -- One of the file extensions supported by the active
backend. Default is ``eps''. Most backend support png, pdf, ps, eps,
and svg.

\end{itemize}

\item[{Returns}] \leavevmode
The figure

\end{description}\end{quote}

\end{fulllineitems}



\subsection{Plotting}
\label{\detokenize{Main:plotting}}\index{plotTs() (in module pyleoclim)}

\begin{fulllineitems}
\phantomsection\label{\detokenize{Main:pyleoclim.plotTs}}\pysiglinewithargsret{\sphinxcode{pyleoclim.}\sphinxbfcode{plotTs}}{\emph{timeseries='`}, \emph{x\_axis='`}, \emph{markersize=50}, \emph{marker='default'}, \emph{saveFig=False}, \emph{dir='`}, \emph{format='eps'}}{}
Plot a single time series.
\begin{quote}\begin{description}
\item[{Parameters}] \leavevmode\begin{itemize}
\item {} 
\sphinxstyleliteralstrong{timeseries} (\sphinxstyleliteralemphasis{A}) -- By default, will prompt the user for one.

\item {} 
\sphinxstyleliteralstrong{x\_axis} (\sphinxstyleliteralemphasis{str}) -- The representation against which to plot the paleo-data.
Options are ``age'', ``year'', and ``depth''. Default is to let the
system choose if only one available or prompt the user.

\item {} 
\sphinxstyleliteralstrong{markersize} (\sphinxstyleliteralemphasis{int}) -- default is 50.

\item {} 
\sphinxstyleliteralstrong{marker} (\sphinxstyleliteralemphasis{str}) -- a string (or list) containing the color and shape of the
marker. Default is by archiveType. Type pyleo.plot\_default to see
the default palette.

\item {} 
\sphinxstyleliteralstrong{saveFig} (\sphinxstyleliteralemphasis{bool}) -- default is to not save the figure

\item {} 
\sphinxstyleliteralstrong{dir} (\sphinxstyleliteralemphasis{str}) -- the full path of the directory in which to save the figure.
If not provided, creates a default folder called `figures' in the
LiPD working directory (lipd.path).

\item {} 
\sphinxstyleliteralstrong{format} (\sphinxstyleliteralemphasis{str}) -- One of the file extensions supported by the active
backend. Default is ``eps''. Most backend support png, pdf, ps, eps,
and svg.

\end{itemize}

\item[{Returns}] \leavevmode
The figure.

\end{description}\end{quote}

\end{fulllineitems}

\index{histTs() (in module pyleoclim)}

\begin{fulllineitems}
\phantomsection\label{\detokenize{Main:pyleoclim.histTs}}\pysiglinewithargsret{\sphinxcode{pyleoclim.}\sphinxbfcode{histTs}}{\emph{timeseries='`}, \emph{bins=None}, \emph{hist=True}, \emph{kde=True}, \emph{rug=False}, \emph{fit=None}, \emph{hist\_kws=\{`label': `Histogram'\}}, \emph{kde\_kws=\{`label': `KDE fit'\}}, \emph{rug\_kws=\{`label': `Rug'\}}, \emph{fit\_kws=\{`label': `Fit'\}}, \emph{color='default'}, \emph{vertical=False}, \emph{norm\_hist=True}, \emph{saveFig=False}, \emph{format='eps'}, \emph{dir='`}}{}
Plot a univariate distribution of the PaleoData values

This function is based on the seaborn displot function, which is
itself a combination of the matplotlib hist function with the
seaborn kdeplot() and rugplot() functions. It can also fit
scipy.stats distributions and plot the estimated PDF over the data.
\begin{quote}\begin{description}
\item[{Parameters}] \leavevmode\begin{itemize}
\item {} 
\sphinxstyleliteralstrong{timeseries} -- A timeseries. By default, will prompt the user for one.

\item {} 
\sphinxstyleliteralstrong{bins} (\sphinxstyleliteralemphasis{int}) -- Specification of hist bins following matplotlib(hist),
or None to use Freedman-Diaconis rule

\item {} 
\sphinxstyleliteralstrong{hist} (\sphinxstyleliteralemphasis{bool}) -- Whether to plot a (normed) histogram

\item {} 
\sphinxstyleliteralstrong{kde} (\sphinxstyleliteralemphasis{bool}) -- Whether to plot a gaussian kernel density estimate

\item {} 
\sphinxstyleliteralstrong{rug} (\sphinxstyleliteralemphasis{bool}) -- Whether to draw a rugplot on the support axis

\item {} 
\sphinxstyleliteralstrong{fit} -- Random variable object. An object with fit method, returning
a tuple that can be passed to a pdf method of positional
arguments following a grid of values to evaluate the pdf on.

\item {} 
\sphinxstyleliteralstrong{kde}\sphinxstyleliteralstrong{, }\sphinxstyleliteralstrong{rug}\sphinxstyleliteralstrong{, }\sphinxstyleliteralstrong{fit\}\_kws} (\sphinxstyleliteralemphasis{\{hist}\sphinxstyleliteralemphasis{,}\sphinxstyleliteralemphasis{}) -- Dictionaries. Keyword arguments for
underlying plotting functions. If modifying the dictionary, make
sure the labels ``hist'', ``kde'', ``rug'' and ``fit'' are still passed.

\item {} 
\sphinxstyleliteralstrong{color} (\sphinxstyleliteralemphasis{str}) -- matplotlib color. Color to plot everything but the
fitted curve in. Default is to use the default paletter for each
archive type.

\item {} 
\sphinxstyleliteralstrong{vertical} (\sphinxstyleliteralemphasis{bool}) -- if True, oberved values are on y-axis.

\item {} 
\sphinxstyleliteralstrong{norm\_hist} (\sphinxstyleliteralemphasis{bool}) -- If True (default), the histrogram height shows
a density rather than a count. This is implied if a KDE or
fitted density is plotted

\item {} 
\sphinxstyleliteralstrong{saveFig} (\sphinxstyleliteralemphasis{bool}) -- default is to not save the figure

\item {} 
\sphinxstyleliteralstrong{dir} (\sphinxstyleliteralemphasis{str}) -- the full path of the directory in which to save the figure.
If not provided, creates a default folder called `figures' in the
LiPD working directory (lipd.path).

\item {} 
\sphinxstyleliteralstrong{format} (\sphinxstyleliteralemphasis{str}) -- One of the file extensions supported by the active
backend. Default is ``eps''. Most backend support png, pdf, ps, eps,
and svg.

\end{itemize}

\end{description}\end{quote}
\begin{description}
\item[{Returns}] \leavevmode
fig - The figure

\end{description}

\end{fulllineitems}



\subsubsection{Summary Plots}
\label{\detokenize{Main:summary-plots}}
Summary plots are a special categories of plots enabled by Pyleoclim.
They allow to plot specific information about a timeseries but are not customizable.
\index{summaryTs() (in module pyleoclim)}

\begin{fulllineitems}
\phantomsection\label{\detokenize{Main:pyleoclim.summaryTs}}\pysiglinewithargsret{\sphinxcode{pyleoclim.}\sphinxbfcode{summaryTs}}{\emph{timeseries='`}, \emph{x\_axis='`}, \emph{saveFig=False}, \emph{dir='`}, \emph{format='eps'}}{}
Basic summary plot

Plots the following information: the time series, a histogram of
the PaleoData\_values, location map, age-depth profile if both are
available from the paleodata, metadata about the record.
\begin{quote}\begin{description}
\item[{Parameters}] \leavevmode\begin{itemize}
\item {} 
\sphinxstyleliteralstrong{timeseries} -- a timeseries object. By default, will prompt for one

\item {} 
\sphinxstyleliteralstrong{x\_axis} (\sphinxstyleliteralemphasis{str}) -- The representation against which to plot the paleo-data.
Options are ``age'', ``year'', and ``depth''. Default is to let the
system choose if only one available or prompt the user.

\item {} 
\sphinxstyleliteralstrong{saveFig} (\sphinxstyleliteralemphasis{bool}) -- default is to not save the figure

\item {} 
\sphinxstyleliteralstrong{dir} (\sphinxstyleliteralemphasis{str}) -- the full path of the directory in which to save the figure.
If not provided, creates a default folder called `figures' in the
LiPD working directory (lipd.path).

\item {} 
\sphinxstyleliteralstrong{format} (\sphinxstyleliteralemphasis{str}) -- One of the file extensions supported by the active
backend. Default is ``eps''. Most backend support png, pdf, ps, eps,
and svg.

\end{itemize}

\item[{Returns}] \leavevmode
The figure

\end{description}\end{quote}

\end{fulllineitems}



\subsection{Statistics}
\label{\detokenize{Main:statistics}}\index{statsTs() (in module pyleoclim)}

\begin{fulllineitems}
\phantomsection\label{\detokenize{Main:pyleoclim.statsTs}}\pysiglinewithargsret{\sphinxcode{pyleoclim.}\sphinxbfcode{statsTs}}{\emph{timeseries='`}}{}
Calculate simple statistics of a timeseries
\begin{quote}\begin{description}
\item[{Parameters}] \leavevmode
\sphinxstyleliteralstrong{timeseries} -- sytem will prompt for one if not given

\item[{Returns}] \leavevmode
the mean, median, min, max, standard deviation and the
inter-quartile range (IQR) of a timeseries.

\end{description}\end{quote}
\paragraph{Examples}

\begin{sphinxVerbatim}[commandchars=\\\{\}]
\PYG{g+gp}{\PYGZgt{}\PYGZgt{}\PYGZgt{} }\PYG{n}{mean}\PYG{p}{,} \PYG{n}{median}\PYG{p}{,} \PYG{n}{min\PYGZus{}}\PYG{p}{,} \PYG{n}{max\PYGZus{}}\PYG{p}{,} \PYG{n}{std}\PYG{p}{,} \PYG{n}{IQR} \PYG{o}{=} \PYG{n}{pyleo}\PYG{o}{.}\PYG{n}{statsTs}\PYG{p}{(}\PYG{n}{timeseries}\PYG{p}{)}
\end{sphinxVerbatim}

\end{fulllineitems}

\index{corrSigTs() (in module pyleoclim)}

\begin{fulllineitems}
\phantomsection\label{\detokenize{Main:pyleoclim.corrSigTs}}\pysiglinewithargsret{\sphinxcode{pyleoclim.}\sphinxbfcode{corrSigTs}}{\emph{timeseries1='`}, \emph{timeseries2='`}, \emph{x\_axis='`}, \emph{interp\_step='`}, \emph{start='`}, \emph{end='`}, \emph{nsim=1000}, \emph{method='isospectral'}, \emph{alpha=0.5}}{}
Estimates the significance of correlations between non IID timeseries.

Function written by. F. Zhu.
\begin{quote}\begin{description}
\item[{Parameters}] \leavevmode\begin{itemize}
\item {} 
\sphinxstyleliteralstrong{timeseries2} (\sphinxstyleliteralemphasis{timeseries1}\sphinxstyleliteralemphasis{,}\sphinxstyleliteralemphasis{}) -- timeseries object. Default is blank.

\item {} 
\sphinxstyleliteralstrong{x-axis} (\sphinxstyleliteralemphasis{str}) -- The representation against which to express the
paleo-data. Options are ``age'', ``year'', and ``depth''.
Default is to let the system choose if only one available
or prompt the user.

\item {} 
\sphinxstyleliteralstrong{interp\_step} (\sphinxstyleliteralemphasis{float}) -- the step size. By default, will prompt the user.

\item {} 
\sphinxstyleliteralstrong{start} (\sphinxstyleliteralemphasis{float}) -- Start time/age/depth. Default is the maximum of
the minima of the two timeseries

\item {} 
\sphinxstyleliteralstrong{end} (\sphinxstyleliteralemphasis{float}) -- End time/age/depth. Default is the minimum of the
maxima of the two timeseries

\item {} 
\sphinxstyleliteralstrong{nsim} (\sphinxstyleliteralemphasis{int}) -- the number of simulations. Default is 1000

\item {} 
\sphinxstyleliteralstrong{method} (\sphinxstyleliteralemphasis{str}) -- 
method use to estimate the correlation and significance.
Available methods include:
\begin{quote}
\begin{itemize}
\item {} 
`ttest': T-test where the degrees of freedom are corrected for

\end{itemize}

the effect of serial correlation
\begin{itemize}
\item {} 
`isopersistant': AR(1) modeling of the two timeseries

\item {} 
`isospectral' (default): phase randomization of original

\end{itemize}

inputs.
\end{quote}

The T-test is parametric test, hence cheap but usually wrong
except in idyllic circumstances.
The others are non-parametric, but their computational
requirements scales with nsim.


\item {} 
\sphinxstyleliteralstrong{alpha} (\sphinxstyleliteralemphasis{float}) -- significance level for critical value estimation. Default is 0.05

\end{itemize}

\item[{Returns}] \leavevmode

r (float) - correlation between the two timeseries

sig (bool) -  Returns True if significant, False otherwise

p (real) - the p-value


\end{description}\end{quote}

\end{fulllineitems}



\subsection{Timeseries}
\label{\detokenize{Main:timeseries}}
Basic manipulations of the timeseries objects.
\index{binTs() (in module pyleoclim)}

\begin{fulllineitems}
\phantomsection\label{\detokenize{Main:pyleoclim.binTs}}\pysiglinewithargsret{\sphinxcode{pyleoclim.}\sphinxbfcode{binTs}}{\emph{timeseries='`}, \emph{x\_axis='`}, \emph{bin\_size='`}, \emph{start='`}, \emph{end='`}}{}
Bin the paleoData values of the timeseries
\begin{quote}\begin{description}
\item[{Parameters}] \leavevmode\begin{itemize}
\item {} 
\sphinxstyleliteralstrong{By default}\sphinxstyleliteralstrong{, }\sphinxstyleliteralstrong{will prompt the user for one.} (\sphinxstyleliteralemphasis{timeseries.}) -- 

\item {} 
\sphinxstyleliteralstrong{x-axis} (\sphinxstyleliteralemphasis{str}) -- The representation against which to plot the paleo-data.
Options are ``age'', ``year'', and ``depth''. Default is to let the
system  choose if only one available or prompt the user.

\item {} 
\sphinxstyleliteralstrong{bin\_size} (\sphinxstyleliteralemphasis{float}) -- the size of the bins to be used. By default,
will prompt for one

\item {} 
\sphinxstyleliteralstrong{start} (\sphinxstyleliteralemphasis{float}) -- Start time/age/depth. Default is the minimum

\item {} 
\sphinxstyleliteralstrong{end} (\sphinxstyleliteralemphasis{float}) -- End time/age/depth. Default is the maximum

\end{itemize}

\item[{Returns}] \leavevmode

binned\_values- the binned output,

bins-  the bins (centered on the median, i.e. the 100-200 bin is 150),

n-  number of data points in each bin,

error- the standard error on the mean in each bin


\end{description}\end{quote}

\end{fulllineitems}

\index{interpTs() (in module pyleoclim)}

\begin{fulllineitems}
\phantomsection\label{\detokenize{Main:pyleoclim.interpTs}}\pysiglinewithargsret{\sphinxcode{pyleoclim.}\sphinxbfcode{interpTs}}{\emph{timeseries='`}, \emph{x\_axis='`}, \emph{interp\_step='`}, \emph{start='`}, \emph{end='`}}{}
Simple linear interpolation

Simple linear interpolation of the data using the numpy.interp method
\begin{quote}\begin{description}
\item[{Parameters}] \leavevmode\begin{itemize}
\item {} 
\sphinxstyleliteralstrong{Default is blank}\sphinxstyleliteralstrong{, }\sphinxstyleliteralstrong{will prompt for it} (\sphinxstyleliteralemphasis{timeseries.}) -- 

\item {} 
\sphinxstyleliteralstrong{x-axis} (\sphinxstyleliteralemphasis{str}) -- The representation against which to plot the paleo-data.
Options are ``age'', ``year'', and ``depth''. Default is to let the
system choose if only one available or prompt the user.

\item {} 
\sphinxstyleliteralstrong{interp\_step} (\sphinxstyleliteralemphasis{float}) -- the step size. By default, will prompt the user.

\item {} 
\sphinxstyleliteralstrong{start} (\sphinxstyleliteralemphasis{float}) -- Start year/age/depth. Default is the minimum

\item {} 
\sphinxstyleliteralstrong{end} (\sphinxstyleliteralemphasis{float}) -- End year/age/depth. Default is the maximum

\end{itemize}

\item[{Returns}] \leavevmode

interp\_age - the interpolated age/year/depth according to the end/start
and time step,

interp\_values - the interpolated values


\end{description}\end{quote}

\end{fulllineitems}

\index{standardizeTs() (in module pyleoclim)}

\begin{fulllineitems}
\phantomsection\label{\detokenize{Main:pyleoclim.standardizeTs}}\pysiglinewithargsret{\sphinxcode{pyleoclim.}\sphinxbfcode{standardizeTs}}{\emph{timeseries='`}, \emph{scale=1}, \emph{ddof=0}, \emph{eps=0.001}}{}
Centers and normalizes the paleoData values of a  given time series.

Constant or nearly constant time series not rescaled.
\begin{quote}\begin{description}
\item[{Parameters}] \leavevmode\begin{itemize}
\item {} 
\sphinxstyleliteralstrong{x} (\sphinxstyleliteralemphasis{array}) -- vector of (real) numbers as a time series, NaNs allowed

\item {} 
\sphinxstyleliteralstrong{scale} (\sphinxstyleliteralemphasis{real}) -- a scale factor used to scale a record to a match a given variance

\item {} 
\sphinxstyleliteralstrong{axis} (\sphinxstyleliteralemphasis{int}\sphinxstyleliteralemphasis{ or }\sphinxstyleliteralemphasis{None}) -- axis along which to operate, if None, compute over the whole array

\item {} 
\sphinxstyleliteralstrong{ddof} (\sphinxstyleliteralemphasis{int}) -- degress of freedom correction in the calculation of the standard deviation

\item {} 
\sphinxstyleliteralstrong{eps} (\sphinxstyleliteralemphasis{real}) -- a threshold to determine if the standard deviation is too close to zero

\end{itemize}

\item[{Returns}] \leavevmode

the standardized time series (z-score), Z = (X - E{[}X{]})/std(X)*scale, NaNs allowed
\begin{itemize}
\item {} 
mu (real): the mean of the original time series, E{[}X{]}

\item {} 
sig (real): the standard deviation of the original time series, std{[}X{]}

\end{itemize}


\item[{Return type}] \leavevmode
\begin{itemize}
\item {} 
z (array)

\end{itemize}


\end{description}\end{quote}
\paragraph{References}
\begin{enumerate}
\item {} 
Tapio Schneider's MATLAB code: \url{http://www.clidyn.ethz.ch/imputation/standardize.m}

\item {} 
The zscore function in SciPy: \url{https://github.com/scipy/scipy/blob/master/scipy/stats/stats.py}

\end{enumerate}

@author: fzhu

\end{fulllineitems}



\subsection{Analysis in the frequency domain}
\label{\detokenize{Main:analysis-in-the-frequency-domain}}\index{wwzTs() (in module pyleoclim)}

\begin{fulllineitems}
\phantomsection\label{\detokenize{Main:pyleoclim.wwzTs}}\pysiglinewithargsret{\sphinxcode{pyleoclim.}\sphinxbfcode{wwzTs}}{\emph{timeseries='`}, \emph{wwz=False}, \emph{psd=True}, \emph{wwz\_default=True}, \emph{psd\_default=True}, \emph{wwaplot\_default=True}, \emph{psdplot\_default=True}, \emph{fig=True}, \emph{saveFig=False}, \emph{dir='`}, \emph{format='eps'}}{}
Weigthed wavelet Z-transform analysis

Wavelet analysis for unevenly spaced data adapted from Foster et al. (1996)
\begin{quote}\begin{description}
\item[{Parameters}] \leavevmode\begin{itemize}
\item {} 
\sphinxstyleliteralstrong{timeseries} (\sphinxstyleliteralemphasis{dict}) -- A LiPD timeseries object (Optional, will prompt for one.)

\item {} 
\sphinxstyleliteralstrong{wwz} (\sphinxstyleliteralemphasis{bool}) -- If True, will perform wavelet analysis

\item {} 
\sphinxstyleliteralstrong{psd} (\sphinxstyleliteralemphasis{bool}) -- If True, will inform the power spectral density of the timeseries

\item {} 
\sphinxstyleliteralstrong{wwz\_default} -- 
If True, will use the following default parameters:
\begin{description}
\item[{wwz\_default = \{`tau':None,'freqs':None,'c':1/(8*np.pi**2),'Neff':3,'nMC':200,}] \leavevmode
`nproc':8,'detrend':'no','method':'Kirchner\_f2py'\}.

\end{description}

Modify the values for specific keys to change the default behavior.


\item {} 
\sphinxstyleliteralstrong{psd\_default} -- 
If True, will use the following default parameters:
\begin{description}
\item[{psd\_default = \{`tau':None,}] \leavevmode
`freqs': None,
`c':1e-3,
`nproc':8,
`nMC':200,
`detrend':'no',
`Neff':3,
`anti\_alias':False,
`avgs':2,
`method':'Kirchner\_f2py'\}

\end{description}

Modify the values for specific keys to change the default behavior.


\item {} 
\sphinxstyleliteralstrong{wwaplot\_default} -- 
If True, will use the following default parameters:
\begin{description}
\item[{wwaplot\_default=\{`Neff':3,}] \leavevmode
`AR1\_q':AR1\_q,
`coi':coi,
`levels':None,
`tick\_range':None,
`yticks':None,
`ylim':None,
`xticks':None,
`xlabels':None,
`figsize':{[}20,8{]},
`clr\_map':'OrRd',
`cbar\_drawedges':False,
`cone\_alpha':0.5,
`plot\_signif':True,
`signif\_style':'contour',
`plot\_cone':True\}

\end{description}

Modify the values for specific keys to change the default behavior.


\item {} 
\sphinxstyleliteralstrong{psdplot\_default} -- 
If True, will use the following default parameters:
\begin{description}
\item[{psdplot\_default=\{`lmstyle':None,}] \leavevmode
`linewidth':None,
`xticks':None,
`xlim':None,
`ylim':None,
`figsize':{[}20,8{]},
`label':'PSD',
`plot\_ar1':True,
`psd\_ar1\_q95':psd\_ar1\_q95,
`psd\_ar1\_color':sns.xkcd\_rgb{[}''pale red''{]}\}

\end{description}

Modify the values for specific keys to change the default behavior.


\item {} 
\sphinxstyleliteralstrong{fig} (\sphinxstyleliteralemphasis{bool}) -- If True, plots the figure

\item {} 
\sphinxstyleliteralstrong{saveFig} (\sphinxstyleliteralemphasis{bool}) -- default is to not save the figure

\item {} 
\sphinxstyleliteralstrong{dir} (\sphinxstyleliteralemphasis{str}) -- the full path of the directory in which to save the figure.
If not provided, creates a default folder called `figures' in the
LiPD working directory (lipd.path).

\item {} 
\sphinxstyleliteralstrong{format} (\sphinxstyleliteralemphasis{str}) -- One of the file extensions supported by the active
backend. Default is ``eps''. Most backend support png, pdf, ps, eps,
and svg.

\end{itemize}

\item[{Returns}] \leavevmode

A dictionary of outputs.
\begin{quote}

For wwz:
\begin{itemize}
\item {} 
wwa (array): The weights wavelet amplitude

\item {} 
AR1\_q (array): AR1 simulations

\item {} 
coi (array): cone of influence

\item {} 
freqs (array): vector for frequencies

\item {} 
tau (array): the evenly-spaced time points, namely the time

\end{itemize}

shift for wavelet analysis.
\begin{itemize}
\item {} 
Neffs (array): The matrix of effective number of points in the

\end{itemize}

time-scale coordinates.
\begin{itemize}
\item {} 
coeff (array): The wavelet transform coefficients

\end{itemize}

For psd:
\begin{itemize}
\item {} 
psd (array): power spectral density

\item {} 
freqs (array): vector of frequency

\item {} 
psd\_ar1\_q95 (array): the 95\% quantile of the psds of AR1 processes

\end{itemize}
\end{quote}

fig: The figure
\paragraph{References}

Foster, G. (1996). Wavelets for period analysis of unevenly
sampled time series. The Astronomical Journal, 112(4), 1709-1729.
\paragraph{Examples}

To run both wwz and psd:

\begin{sphinxVerbatim}[commandchars=\\\{\}]
\PYG{g+gp}{\PYGZgt{}\PYGZgt{}\PYGZgt{} }\PYG{n}{dict\PYGZus{}out}\PYG{p}{,} \PYG{n}{fig} \PYG{o}{=} \PYG{n}{pyleoclim}\PYG{o}{.}\PYG{n}{wwzTs}\PYG{p}{(}\PYG{n}{wwz}\PYG{o}{=}\PYG{k+kc}{True}\PYG{p}{)}
\end{sphinxVerbatim}

Note: This will return a single figure with wwa and psd

To change a default behavior:

\begin{sphinxVerbatim}[commandchars=\\\{\}]
\PYG{g+gp}{\PYGZgt{}\PYGZgt{}\PYGZgt{} }\PYG{n}{dict\PYGZus{}out}\PYG{p}{,} \PYG{n}{fig} \PYG{o}{=} \PYG{n}{pyleoclim}\PYG{o}{.}\PYG{n}{wwzTs}\PYG{p}{(}\PYG{n}{psd\PYGZus{}default} \PYG{o}{=} \PYG{p}{\PYGZob{}}\PYG{l+s+s1}{\PYGZsq{}}\PYG{l+s+s1}{nMC}\PYG{l+s+s1}{\PYGZsq{}}\PYG{p}{:}\PYG{l+m+mi}{1000}\PYG{p}{\PYGZcb{}}\PYG{p}{)}
\end{sphinxVerbatim}


\item[{Return type}] \leavevmode
dict\_out (dict)

\end{description}\end{quote}

\end{fulllineitems}



\section{Using Pyleoclim without a LiPD file}
\label{\detokenize{Main:using-pyleoclim-without-a-lipd-file}}
The Pyleoclim modules can be called separately so the main functions can be used without a timeseries objects.
The following modules are available:
\begin{itemize}
\item {} 
pyleoclim.Map: mapping functions

\item {} 
pyleoclim.Plot: plotting functions

\item {} 
pyleoclim.Stats: statistics (including correlation)

\item {} 
pyleoclim.Timeseries: binning, interpolating

\item {} 
pyleoclim.Spectral: analysis in the frequency domain

\end{itemize}

In addition, the pyleoclim.LipdUtils module allows the basic manipulation of LiPD files and the pyleoclim.SummaryPlots module
contains some functions to extract information about a timeseries object.


\chapter{Mapping Functions}
\label{\detokenize{Map:mapping-functions}}\label{\detokenize{Map::doc}}
This module uses the Basemap package for mapping.
\index{mapAll() (in module pyleoclim.Map)}

\begin{fulllineitems}
\phantomsection\label{\detokenize{Map:pyleoclim.Map.mapAll}}\pysiglinewithargsret{\sphinxcode{pyleoclim.Map.}\sphinxbfcode{mapAll}}{\emph{lat}, \emph{lon}, \emph{criteria}, \emph{projection='robin'}, \emph{lat\_0='`}, \emph{lon\_0='`}, \emph{llcrnrlat=-90}, \emph{urcrnrlat=90}, \emph{llcrnrlon=-180}, \emph{urcrnrlon=180}, \emph{countries=False}, \emph{counties=False}, \emph{rivers=False}, \emph{states=False}, \emph{background='none'}, \emph{scale=0.5}, \emph{palette='`}, \emph{markersize=50}}{}
Map the location of all lat/lon according to some criteria

Map the location of all lat/lon according to some criteria. The choice of
plotting color/marker is passed through palette according to unique
criteria (e.g., record name, archive type, proxy observation type).
\begin{quote}\begin{description}
\item[{Parameters}] \leavevmode\begin{itemize}
\item {} 
\sphinxstyleliteralstrong{lat} (\sphinxstyleliteralemphasis{list}) -- a list of latitude.

\item {} 
\sphinxstyleliteralstrong{lon} (\sphinxstyleliteralemphasis{list}) -- a list of longitude.

\item {} 
\sphinxstyleliteralstrong{criteria} (\sphinxstyleliteralemphasis{list}) -- a list of criteria for plotting purposes. For instance,
a map by the types of archive present in the dataset or proxy
observations.

\item {} 
\sphinxstyleliteralstrong{projection} (\sphinxstyleliteralemphasis{string}) -- the map projection. Refers to the Basemap
documentation for a list of available projections. Only projections
supporting setting the map center with a single lat/lon or with
the coordinates of the rectangle are currently supported.
Default is to use a Robinson projection.

\item {} 
\sphinxstyleliteralstrong{lon\_0} (\sphinxstyleliteralemphasis{lat\_0}\sphinxstyleliteralemphasis{,}\sphinxstyleliteralemphasis{}) -- the center coordinates for the map. Default is
mean latitude/longitude in the list.
If the chosen projection doesn't support it, Basemap will
ignore the given values.

\item {} 
\sphinxstyleliteralstrong{urcrnrlat}\sphinxstyleliteralstrong{, }\sphinxstyleliteralstrong{llcrnrlon}\sphinxstyleliteralstrong{, }\sphinxstyleliteralstrong{urcrnrlon} (\sphinxstyleliteralemphasis{llcrnrlat}\sphinxstyleliteralemphasis{,}\sphinxstyleliteralemphasis{}) -- The coordinates
of the two opposite corners of the rectangle.

\item {} 
\sphinxstyleliteralstrong{countries} (\sphinxstyleliteralemphasis{bool}) -- Draws the countries border. Defaults is off (False).

\item {} 
\sphinxstyleliteralstrong{counties} (\sphinxstyleliteralemphasis{bool}) -- Draws the USA counties. Default is off (False).

\item {} 
\sphinxstyleliteralstrong{rivers} (\sphinxstyleliteralemphasis{bool}) -- Draws the rivers. Default is off (False).

\item {} 
\sphinxstyleliteralstrong{states} (\sphinxstyleliteralemphasis{bool}) -- Draws the American and Australian states borders.
Default is off (False).

\item {} 
\sphinxstyleliteralstrong{background} (\sphinxstyleliteralemphasis{string}) -- Plots one of the following images on the map:
bluemarble, etopo, shadedrelief, or none (filled continents).
Default is none.

\item {} 
\sphinxstyleliteralstrong{scale} (\sphinxstyleliteralemphasis{float}) -- Useful to downgrade the original image resolution to
speed up the process. Default is 0.5.

\item {} 
\sphinxstyleliteralstrong{palette} (\sphinxstyleliteralemphasis{dict}) -- A dictionary of plotting color/marker by criteria. The
keys should correspond to \sphinxstylestrong{*unique*} criteria with a list of
associated values. The list should be in the format
{[}'color', `marker'{]}.

\item {} 
\sphinxstyleliteralstrong{markersize} (\sphinxstyleliteralemphasis{int}) -- The size of the marker.

\end{itemize}

\item[{Returns}] \leavevmode
The figure

\end{description}\end{quote}

\end{fulllineitems}

\index{mapOne() (in module pyleoclim.Map)}

\begin{fulllineitems}
\phantomsection\label{\detokenize{Map:pyleoclim.Map.mapOne}}\pysiglinewithargsret{\sphinxcode{pyleoclim.Map.}\sphinxbfcode{mapOne}}{\emph{lat}, \emph{lon}, \emph{projection='ortho'}, \emph{lat\_0='`}, \emph{lon\_0='`}, \emph{llcrnrlat=-90}, \emph{urcrnrlat=90}, \emph{llcrnrlon=-180}, \emph{urcrnrlon=180}, \emph{countries=True}, \emph{counties=False}, \emph{rivers=False}, \emph{states=False}, \emph{background='shadedrelief'}, \emph{scale=0.5}, \emph{markersize=50}, \emph{marker='ro'}}{}
Map one location on the globe
\begin{quote}\begin{description}
\item[{Parameters}] \leavevmode\begin{itemize}
\item {} 
\sphinxstyleliteralstrong{lat} (\sphinxstyleliteralemphasis{float}) -- a float number representing latitude

\item {} 
\sphinxstyleliteralstrong{lon} (\sphinxstyleliteralemphasis{float}) -- a float number representing longitude

\item {} 
\sphinxstyleliteralstrong{projection} (\sphinxstyleliteralemphasis{string}) -- the map projection. Refers to the Basemap
documentation for a list of available projections. Only projections
supporting setting the map center with a single lat/lon or with
the coordinates of the rectangle are currently supported.
Default is to use a Robinson projection.

\item {} 
\sphinxstyleliteralstrong{lon\_0} (\sphinxstyleliteralemphasis{lat\_0}\sphinxstyleliteralemphasis{,}\sphinxstyleliteralemphasis{}) -- the center coordinates for the map. Default is
mean latitude/longitude in the list.
If the chosen projection doesn't support it, Basemap will
ignore the given values.

\item {} 
\sphinxstyleliteralstrong{urcrnrlat}\sphinxstyleliteralstrong{, }\sphinxstyleliteralstrong{llcrnrlon}\sphinxstyleliteralstrong{, }\sphinxstyleliteralstrong{urcrnrlon} (\sphinxstyleliteralemphasis{llcrnrlat}\sphinxstyleliteralemphasis{,}\sphinxstyleliteralemphasis{}) -- The coordinates
of the two opposite corners of the rectangle.

\item {} 
\sphinxstyleliteralstrong{countries} (\sphinxstyleliteralemphasis{bool}) -- Draws the countries border. Defaults is off (False).

\item {} 
\sphinxstyleliteralstrong{counties} (\sphinxstyleliteralemphasis{bool}) -- Draws the USA counties. Default is off (False).

\item {} 
\sphinxstyleliteralstrong{rivers} (\sphinxstyleliteralemphasis{bool}) -- Draws the rivers. Default is off (False).

\item {} 
\sphinxstyleliteralstrong{states} (\sphinxstyleliteralemphasis{bool}) -- Draws the American and Australian states borders.
Default is off (False).

\item {} 
\sphinxstyleliteralstrong{background} (\sphinxstyleliteralemphasis{string}) -- Plots one of the following images on the map:
bluemarble, etopo, shadedrelief, or none (filled continents).
Default is none.

\item {} 
\sphinxstyleliteralstrong{scale} (\sphinxstyleliteralemphasis{float}) -- Useful to downgrade the original image resolution to
speed up the process. Default is 0.5.

\item {} 
\sphinxstyleliteralstrong{markersize} (\sphinxstyleliteralemphasis{int}) -- The size of the marker.

\item {} 
\sphinxstyleliteralstrong{marker} (\sphinxstyleliteralemphasis{str}\sphinxstyleliteralemphasis{ or }\sphinxstyleliteralemphasis{list}) -- color and type of marker.

\end{itemize}

\end{description}\end{quote}

\end{fulllineitems}



\chapter{Plotting Functions}
\label{\detokenize{Plot:plotting-functions}}\label{\detokenize{Plot::doc}}\index{plot() (in module pyleoclim.Plot)}

\begin{fulllineitems}
\phantomsection\label{\detokenize{Plot:pyleoclim.Plot.plot}}\pysiglinewithargsret{\sphinxcode{pyleoclim.Plot.}\sphinxbfcode{plot}}{\emph{x}, \emph{y}, \emph{markersize=50}, \emph{marker='ro'}, \emph{x\_label='`}, \emph{y\_label='`}, \emph{title='`}}{}
Make a 2-D plot
\begin{quote}\begin{description}
\item[{Parameters}] \leavevmode\begin{itemize}
\item {} 
\sphinxstyleliteralstrong{x} (\sphinxstyleliteralemphasis{numpy array}) -- a 1xn numpy array of values for the x-axis

\item {} 
\sphinxstyleliteralstrong{y} (\sphinxstyleliteralemphasis{numpy array}) -- a 1xn numpy array for the y-axis

\item {} 
\sphinxstyleliteralstrong{markersize} (\sphinxstyleliteralemphasis{int}) -- the size of the marker

\item {} 
\sphinxstyleliteralstrong{marker} (\sphinxstyleliteralemphasis{string}\sphinxstyleliteralemphasis{ or }\sphinxstyleliteralemphasis{list}) -- color and shape of the marker

\item {} 
\sphinxstyleliteralstrong{x\_axis\_label} (\sphinxstyleliteralemphasis{str}) -- the label for the x-axis

\item {} 
\sphinxstyleliteralstrong{y\_axis\_label} (\sphinxstyleliteralemphasis{str}) -- the label for the y-axis

\item {} 
\sphinxstyleliteralstrong{title} (\sphinxstyleliteralemphasis{str}) -- the title for the plot

\end{itemize}

\item[{Returns}] \leavevmode
The figure

\end{description}\end{quote}

\end{fulllineitems}

\index{plot\_hist() (in module pyleoclim.Plot)}

\begin{fulllineitems}
\phantomsection\label{\detokenize{Plot:pyleoclim.Plot.plot_hist}}\pysiglinewithargsret{\sphinxcode{pyleoclim.Plot.}\sphinxbfcode{plot\_hist}}{\emph{y}, \emph{bins=None}, \emph{hist=True}, \emph{label='`}, \emph{kde=True}, \emph{rug=False}, \emph{fit=None}, \emph{hist\_kws=\{`label': `Histogram'\}}, \emph{kde\_kws=\{`label': `KDE fit'\}}, \emph{rug\_kws=\{`label': `rug'\}}, \emph{fit\_kws=\{`label': `fit'\}}, \emph{color=`0.7'}, \emph{vertical=False}, \emph{norm\_hist=True}}{}
Plot a univariate distribution of the PaleoData values

This function is based on the seaborn displot function, which is
itself a combination of the matplotlib hist function with the
seaborn kdeplot() and rugplot() functions. It can also fit
scipy.stats distributions and plot the estimated PDF over the data.
\begin{quote}\begin{description}
\item[{Parameters}] \leavevmode\begin{itemize}
\item {} 
\sphinxstyleliteralstrong{y} (\sphinxstyleliteralemphasis{array}) -- nx1 numpy array. No missing values allowed

\item {} 
\sphinxstyleliteralstrong{bins} (\sphinxstyleliteralemphasis{int}) -- Specification of hist bins following matplotlib(hist),
or None to use Freedman-Diaconis rule

\item {} 
\sphinxstyleliteralstrong{hist} (\sphinxstyleliteralemphasis{bool}) -- Whether to plot a (normed) histogram

\item {} 
\sphinxstyleliteralstrong{label} (\sphinxstyleliteralemphasis{str}) -- The label for the axis

\item {} 
\sphinxstyleliteralstrong{kde} (\sphinxstyleliteralemphasis{bool}) -- Whether to plot a gaussian kernel density estimate

\item {} 
\sphinxstyleliteralstrong{rug} (\sphinxstyleliteralemphasis{bool}) -- Whether to draw a rugplot on the support axis

\item {} 
\sphinxstyleliteralstrong{fit} -- Random variable object. An object with fit method, returning
a tuple that can be passed to a pdf method of positional
arguments following a grid of values to evaluate the pdf on.

\item {} 
\sphinxstyleliteralstrong{kde}\sphinxstyleliteralstrong{, }\sphinxstyleliteralstrong{rug}\sphinxstyleliteralstrong{, }\sphinxstyleliteralstrong{fit\}\_kws} (\sphinxstyleliteralemphasis{\{hist}\sphinxstyleliteralemphasis{,}\sphinxstyleliteralemphasis{}) -- Dictionaries. Keyword arguments for
underlying plotting functions. If modifying the dictionary, make
sure the labels ``hist'', ``kde'', ``rug'' and ``fit'' are still passed.

\item {} 
\sphinxstyleliteralstrong{color} (\sphinxstyleliteralemphasis{str}) -- matplotlib color. Color to plot everything but the
fitted curve in.

\item {} 
\sphinxstyleliteralstrong{vertical} (\sphinxstyleliteralemphasis{bool}) -- if True, oberved values are on y-axis.

\item {} 
\sphinxstyleliteralstrong{norm\_hist} (\sphinxstyleliteralemphasis{bool}) -- If True (default), the histrogram height shows
a density rather than a count. This is implied if a KDE or
fitted density is plotted

\end{itemize}

\end{description}\end{quote}
\begin{description}
\item[{Returns}] \leavevmode
fig - The figure

\end{description}

\end{fulllineitems}



\chapter{Statistics Functions}
\label{\detokenize{Stats::doc}}\label{\detokenize{Stats:statistics-functions}}\index{simpleStats() (in module pyleoclim.Stats)}

\begin{fulllineitems}
\phantomsection\label{\detokenize{Stats:pyleoclim.Stats.simpleStats}}\pysiglinewithargsret{\sphinxcode{pyleoclim.Stats.}\sphinxbfcode{simpleStats}}{\emph{y}, \emph{axis=None}}{}
Computes simple statistics

Computes the mean, median, min, max, standard deviation, and interquartile
range of a numpy array y.
\begin{quote}\begin{description}
\item[{Parameters}] \leavevmode\begin{itemize}
\item {} 
\sphinxstyleliteralstrong{y} (\sphinxstyleliteralemphasis{array}) -- A Numpy array

\item {} 
\sphinxstyleliteralstrong{axis} (\sphinxstyleliteralemphasis{int}\sphinxstyleliteralemphasis{, }\sphinxstyleliteralemphasis{typle of ints}) -- Optional. Axis or Axes along which the means
are computed, the default is to compute the mean of the flattened
array. If a tuple of ints, performed over multiple axes

\end{itemize}

\item[{Returns}] \leavevmode
The mean, median, min, max, standard deviation and IQR by columns

\end{description}\end{quote}

\end{fulllineitems}

\index{corrsig() (in module pyleoclim.Stats)}

\begin{fulllineitems}
\phantomsection\label{\detokenize{Stats:pyleoclim.Stats.corrsig}}\pysiglinewithargsret{\sphinxcode{pyleoclim.Stats.}\sphinxbfcode{corrsig}}{\emph{y1}, \emph{y2}, \emph{nsim=1000}, \emph{method='isospectral'}, \emph{alpha=0.5}}{}~\begin{description}
\item[{Estimates the significance of correlations between non IID time series by 3 independent methods:}] \leavevmode\begin{enumerate}
\item {} 
`ttest': T-test where d.o.f are corrected for the effect of serial correlation

\item {} 
`isopersistent': AR(1) modeling of x and y.

\end{enumerate}

3) `isospectral': phase randomization of original inputs. (default)
The T-test is parametric test, hence cheap but usually wrong except in idyllic circumstances.
The others are non-parametric, but their computational requirements scales with nsim.

\end{description}
\begin{quote}\begin{description}
\item[{Parameters}] \leavevmode\begin{itemize}
\item {} 
\sphinxstyleliteralstrong{y2} (\sphinxstyleliteralemphasis{y1}\sphinxstyleliteralemphasis{,}\sphinxstyleliteralemphasis{}) -- 

\item {} 
\sphinxstyleliteralstrong{nsim} (\sphinxstyleliteralemphasis{int}) -- 

\item {} 
\sphinxstyleliteralstrong{method} (\sphinxstyleliteralemphasis{str}) -- 

\item {} 
\sphinxstyleliteralstrong{alpha} (\sphinxstyleliteralemphasis{float}) -- 

\end{itemize}

\item[{Returns}] \leavevmode

correlation between x and y

signif (int): true  if significant; false otherwise

p (real): Fraction of time series with higher correlation coefficents than observed (approximates the p-value).
\begin{quote}

Note that signif = True if and only if p \textless{}= alpha.
\end{quote}


\item[{Return type}] \leavevmode
r (real)

\end{description}\end{quote}

\end{fulllineitems}



\chapter{Timeseries Functions}
\label{\detokenize{Timeseries::doc}}\label{\detokenize{Timeseries:timeseries-functions}}
This module allows the manipulation of timeseries.
\index{bin() (in module pyleoclim.Timeseries)}

\begin{fulllineitems}
\phantomsection\label{\detokenize{Timeseries:pyleoclim.Timeseries.bin}}\pysiglinewithargsret{\sphinxcode{pyleoclim.Timeseries.}\sphinxbfcode{bin}}{\emph{x}, \emph{y}, \emph{bin\_size='`}, \emph{start='`}, \emph{end='`}}{}
Bin the values
\begin{quote}\begin{description}
\item[{Parameters}] \leavevmode\begin{itemize}
\item {} 
\sphinxstyleliteralstrong{x} (\sphinxstyleliteralemphasis{array}) -- the x-axis series.

\item {} 
\sphinxstyleliteralstrong{y} (\sphinxstyleliteralemphasis{array}) -- the y-axis series.

\item {} 
\sphinxstyleliteralstrong{bin\_size} (\sphinxstyleliteralemphasis{float}) -- The size of the bins. Default is the average resolution

\item {} 
\sphinxstyleliteralstrong{start} (\sphinxstyleliteralemphasis{float}) -- Where/when to start binning. Default is the minimum

\item {} 
\sphinxstyleliteralstrong{end} (\sphinxstyleliteralemphasis{float}) -- When/where to stop binning. Defulat is the maximum

\end{itemize}

\item[{Returns}] \leavevmode

binned\_values - the binned output

bins - the bins (centered on the median, i.e., the 100-200 bin is 150)

n - number of data points in each bin

error -  the standard error on the mean in each bin


\end{description}\end{quote}

\end{fulllineitems}

\index{interp() (in module pyleoclim.Timeseries)}

\begin{fulllineitems}
\phantomsection\label{\detokenize{Timeseries:pyleoclim.Timeseries.interp}}\pysiglinewithargsret{\sphinxcode{pyleoclim.Timeseries.}\sphinxbfcode{interp}}{\emph{x}, \emph{y}, \emph{interp\_step='`}, \emph{start='`}, \emph{end='`}}{}
Linear interpolation onto a new x-axis
\begin{quote}\begin{description}
\item[{Parameters}] \leavevmode\begin{itemize}
\item {} 
\sphinxstyleliteralstrong{x} (\sphinxstyleliteralemphasis{array}) -- the x-axis

\item {} 
\sphinxstyleliteralstrong{y} (\sphinxstyleliteralemphasis{array}) -- the y-axis

\item {} 
\sphinxstyleliteralstrong{interp\_step} (\sphinxstyleliteralemphasis{float}) -- the interpolation step. Default is mean resolution.

\item {} 
\sphinxstyleliteralstrong{start} (\sphinxstyleliteralemphasis{float}) -- where/when to start the interpolation. Default is min..

\item {} 
\sphinxstyleliteralstrong{end} (\sphinxstyleliteralemphasis{float}) -- where/when to stop the interpolation. Defaul is max.

\end{itemize}

\item[{Returns}] \leavevmode

xi - the interpolated x-axis

interp\_values - the interpolated values


\end{description}\end{quote}

\end{fulllineitems}

\index{onCommonAxis() (in module pyleoclim.Timeseries)}

\begin{fulllineitems}
\phantomsection\label{\detokenize{Timeseries:pyleoclim.Timeseries.onCommonAxis}}\pysiglinewithargsret{\sphinxcode{pyleoclim.Timeseries.}\sphinxbfcode{onCommonAxis}}{\emph{x1}, \emph{y1}, \emph{x2}, \emph{y2}, \emph{interp\_step='`}, \emph{start='`}, \emph{end='`}}{}
Places two timeseries on a common axis
\begin{quote}\begin{description}
\item[{Parameters}] \leavevmode\begin{itemize}
\item {} 
\sphinxstyleliteralstrong{x1} (\sphinxstyleliteralemphasis{array}) -- x-axis values of the first timeseries

\item {} 
\sphinxstyleliteralstrong{y1} (\sphinxstyleliteralemphasis{array}) -- y-axis values of the first timeseries

\item {} 
\sphinxstyleliteralstrong{x2} (\sphinxstyleliteralemphasis{array}) -- x-axis values of the second timeseries

\item {} 
\sphinxstyleliteralstrong{y2} (\sphinxstyleliteralemphasis{array}) -- y-axis values of the second timeseries

\item {} 
\sphinxstyleliteralstrong{interp\_step} (\sphinxstyleliteralemphasis{float}) -- The interpolation step. Default is mean resolution
of lowest resolution series

\item {} 
\sphinxstyleliteralstrong{start} (\sphinxstyleliteralemphasis{float}) -- where/when to start. Default is the maximum of the minima of
the two timeseries

\item {} 
\sphinxstyleliteralstrong{end} (\sphinxstyleliteralemphasis{float}) -- Where/when to end. Default is the minimum of the maxima of
the two timeseries

\end{itemize}

\item[{Returns}] \leavevmode

xi -  the interpolated x-axis

interp\_values1 -  the interpolated y-values for the first timeseries
interp\_values2 - the intespolated y-values for the second timeseries


\end{description}\end{quote}

\end{fulllineitems}

\index{standardize() (in module pyleoclim.Timeseries)}

\begin{fulllineitems}
\phantomsection\label{\detokenize{Timeseries:pyleoclim.Timeseries.standardize}}\pysiglinewithargsret{\sphinxcode{pyleoclim.Timeseries.}\sphinxbfcode{standardize}}{\emph{x}, \emph{scale=1}, \emph{axis=0}, \emph{ddof=0}, \emph{eps=0.001}}{}
Centers and normalizes a given time series. Constant or nearly constant time series not rescaled.
\begin{quote}\begin{description}
\item[{Parameters}] \leavevmode\begin{itemize}
\item {} 
\sphinxstyleliteralstrong{x} (\sphinxstyleliteralemphasis{array}) -- vector of (real) numbers as a time series, NaNs allowed

\item {} 
\sphinxstyleliteralstrong{scale} (\sphinxstyleliteralemphasis{real}) -- a scale factor used to scale a record to a match a given variance

\item {} 
\sphinxstyleliteralstrong{axis} (\sphinxstyleliteralemphasis{int}\sphinxstyleliteralemphasis{ or }\sphinxstyleliteralemphasis{None}) -- axis along which to operate, if None, compute over the whole array

\item {} 
\sphinxstyleliteralstrong{ddof} (\sphinxstyleliteralemphasis{int}) -- degress of freedom correction in the calculation of the standard deviation

\item {} 
\sphinxstyleliteralstrong{eps} (\sphinxstyleliteralemphasis{real}) -- a threshold to determine if the standard deviation is too close to zero

\end{itemize}

\item[{Returns}] \leavevmode
the standardized time series (z-score), Z = (X - E{[}X{]})/std(X)*scale, NaNs allowed
mu (real): the mean of the original time series, E{[}X{]}
sig (real): the standard deviation of the original time series, std{[}X{]}

\item[{Return type}] \leavevmode
z (array)

\end{description}\end{quote}
\paragraph{References}
\begin{enumerate}
\item {} 
Tapio Schneider's MATLAB code: \url{http://www.clidyn.ethz.ch/imputation/standardize.m}

\item {} 
The zscore function in SciPy: \url{https://github.com/scipy/scipy/blob/master/scipy/stats/stats.py}

\end{enumerate}

@author: fzhu

\end{fulllineitems}



\chapter{LiPD Utilities}
\label{\detokenize{LipdUtils:lipd-utilities}}\label{\detokenize{LipdUtils::doc}}
This modules allow basic manipulation of LiPD files


\section{Creating Directories and saving}
\label{\detokenize{LipdUtils:creating-directories-and-saving}}\index{createDir() (in module pyleoclim.LipdUtils)}

\begin{fulllineitems}
\phantomsection\label{\detokenize{LipdUtils:pyleoclim.LipdUtils.createDir}}\pysiglinewithargsret{\sphinxcode{pyleoclim.LipdUtils.}\sphinxbfcode{createDir}}{\emph{path}, \emph{foldername}}{}
Create a new folder in a working directory

Create a new folder in a working directory to save outputs from Pyleoclim.
\begin{quote}\begin{description}
\item[{Parameters}] \leavevmode\begin{itemize}
\item {} 
\sphinxstyleliteralstrong{path} (\sphinxstyleliteralemphasis{str}) -- the path to the new folder.

\item {} 
\sphinxstyleliteralstrong{foldername} (\sphinxstyleliteralemphasis{str}) -- the name of the folder to be created

\end{itemize}

\item[{Returns}] \leavevmode
newdir - the full path to the new directory

\end{description}\end{quote}

\end{fulllineitems}

\index{saveFigure() (in module pyleoclim.LipdUtils)}

\begin{fulllineitems}
\phantomsection\label{\detokenize{LipdUtils:pyleoclim.LipdUtils.saveFigure}}\pysiglinewithargsret{\sphinxcode{pyleoclim.LipdUtils.}\sphinxbfcode{saveFigure}}{\emph{name}, \emph{format='eps'}, \emph{dir='`}}{}
Save a figure

Save the figure in the directory. If not given, creates a folder in the
current working directory.
\begin{quote}\begin{description}
\item[{Parameters}] \leavevmode\begin{itemize}
\item {} 
\sphinxstyleliteralstrong{name} (\sphinxstyleliteralemphasis{str}) -- name of the file

\item {} 
\sphinxstyleliteralstrong{format} (\sphinxstyleliteralemphasis{str}) -- One of the file extensions supported by the active
backend. Default is ``eps''. Most backend support png, pdf, ps, eps,
and svg.

\item {} 
\sphinxstyleliteralstrong{dir} (\sphinxstyleliteralemphasis{str}) -- the name of the folder in the LiPD working directory.
If not provided, creates a default folder called `figures'.

\end{itemize}

\end{description}\end{quote}

\end{fulllineitems}



\section{LiPD files}
\label{\detokenize{LipdUtils:lipd-files}}\index{enumerateLipds() (in module pyleoclim.LipdUtils)}

\begin{fulllineitems}
\phantomsection\label{\detokenize{LipdUtils:pyleoclim.LipdUtils.enumerateLipds}}\pysiglinewithargsret{\sphinxcode{pyleoclim.LipdUtils.}\sphinxbfcode{enumerateLipds}}{}{}
Enumerate the LiPD files loaded in the workspace

\end{fulllineitems}

\index{promptForLipd() (in module pyleoclim.LipdUtils)}

\begin{fulllineitems}
\phantomsection\label{\detokenize{LipdUtils:pyleoclim.LipdUtils.promptForLipd}}\pysiglinewithargsret{\sphinxcode{pyleoclim.LipdUtils.}\sphinxbfcode{promptForLipd}}{}{}
Prompt for a LiPD file

Ask the user to select a LiPD file from a list
Use this function in conjunction with enumerateLipds()
\begin{quote}\begin{description}
\item[{Returns}] \leavevmode
The index of the LiPD file

\end{description}\end{quote}

\end{fulllineitems}



\section{Handling Variables}
\label{\detokenize{LipdUtils:handling-variables}}\index{promptForVariable() (in module pyleoclim.LipdUtils)}

\begin{fulllineitems}
\phantomsection\label{\detokenize{LipdUtils:pyleoclim.LipdUtils.promptForVariable}}\pysiglinewithargsret{\sphinxcode{pyleoclim.LipdUtils.}\sphinxbfcode{promptForVariable}}{}{}
Prompt for a specific variable

Ask the user to select the variable they are interested in.
Use this function in conjunction with readHeaders() or getTSO()
\begin{quote}\begin{description}
\item[{Returns}] \leavevmode
The index of the variable

\end{description}\end{quote}

\end{fulllineitems}

\index{xAxisTs() (in module pyleoclim.LipdUtils)}

\begin{fulllineitems}
\phantomsection\label{\detokenize{LipdUtils:pyleoclim.LipdUtils.xAxisTs}}\pysiglinewithargsret{\sphinxcode{pyleoclim.LipdUtils.}\sphinxbfcode{xAxisTs}}{\emph{timeseries}}{}
Prompt the user to choose a x-axis representation for the timeseries.
\begin{quote}\begin{description}
\item[{Parameters}] \leavevmode
\sphinxstyleliteralstrong{timeseries} -- a timeseries object

\item[{Returns}] \leavevmode

x\_axis - the values for the x-axis representation,

label - returns either ``age'', ``year'', or ``depth''


\end{description}\end{quote}

\end{fulllineitems}

\index{checkXaxis() (in module pyleoclim.LipdUtils)}

\begin{fulllineitems}
\phantomsection\label{\detokenize{LipdUtils:pyleoclim.LipdUtils.checkXaxis}}\pysiglinewithargsret{\sphinxcode{pyleoclim.LipdUtils.}\sphinxbfcode{checkXaxis}}{\emph{timeseries}, \emph{x\_axis='`}}{}
Check that a x-axis is present for the timeseries
\begin{quote}\begin{description}
\item[{Parameters}] \leavevmode\begin{itemize}
\item {} 
\sphinxstyleliteralstrong{timeseries} -- a timeseries

\item {} 
\sphinxstyleliteralstrong{x\_axis} (\sphinxstyleliteralemphasis{str}) -- the x-axis representation, either depth, age or year

\end{itemize}

\item[{Returns}] \leavevmode

x - the values for the x-axis representation,

label - returns either ``age'', ``year'', or ``depth''


\end{description}\end{quote}

\end{fulllineitems}



\section{Handling timeseries objects}
\label{\detokenize{LipdUtils:handling-timeseries-objects}}\index{enumerateTs() (in module pyleoclim.LipdUtils)}

\begin{fulllineitems}
\phantomsection\label{\detokenize{LipdUtils:pyleoclim.LipdUtils.enumerateTs}}\pysiglinewithargsret{\sphinxcode{pyleoclim.LipdUtils.}\sphinxbfcode{enumerateTs}}{\emph{timeseries\_list}}{}
Enumerate the available time series objects
\begin{quote}\begin{description}
\item[{Parameters}] \leavevmode
\sphinxstyleliteralstrong{timeseries\_list} -- a  list of available timeseries objects.
To use the timeseries loaded upon initiation of the
pyleoclim package, use pyleo.time\_series.

\end{description}\end{quote}

\end{fulllineitems}

\index{getTs() (in module pyleoclim.LipdUtils)}

\begin{fulllineitems}
\phantomsection\label{\detokenize{LipdUtils:pyleoclim.LipdUtils.getTs}}\pysiglinewithargsret{\sphinxcode{pyleoclim.LipdUtils.}\sphinxbfcode{getTs}}{\emph{timeseries\_list}}{}
Get a specific timeseries object from a dictionary of timeseries
\begin{quote}\begin{description}
\item[{Parameters}] \leavevmode
\sphinxstyleliteralstrong{timeseries\_list} -- a  list of available timeseries objects.
To use the timeseries loaded upon initiation of the
pyleoclim package, use pyleo.time\_series.

\item[{Returns}] \leavevmode
A single timeseries object

\end{description}\end{quote}

\end{fulllineitems}



\section{Linking LiPDs to the LinkedEarth Ontology}
\label{\detokenize{LipdUtils:linking-lipds-to-the-linkedearth-ontology}}\index{LipdToOntology() (in module pyleoclim.LipdUtils)}

\begin{fulllineitems}
\phantomsection\label{\detokenize{LipdUtils:pyleoclim.LipdUtils.LipdToOntology}}\pysiglinewithargsret{\sphinxcode{pyleoclim.LipdUtils.}\sphinxbfcode{LipdToOntology}}{\emph{archiveType}}{}
standardize archiveType

Transform the archiveType from their LiPD name to their ontology counterpart
\begin{quote}\begin{description}
\item[{Parameters}] \leavevmode
\sphinxstyleliteralstrong{archiveType} (\sphinxstyleliteralemphasis{STR}) -- name of the archiveType from the LiPD file

\item[{Returns}] \leavevmode
archiveType according to the ontology

\end{description}\end{quote}

\end{fulllineitems}



\chapter{Summary Plots}
\label{\detokenize{SummaryPlots:summary-plots}}\label{\detokenize{SummaryPlots::doc}}
This module handles some basic return for the summary plots.

\sphinxstylestrong{Requires} a LiPD file
\index{getMetadata() (in module pyleoclim.SummaryPlots)}

\begin{fulllineitems}
\phantomsection\label{\detokenize{SummaryPlots:pyleoclim.SummaryPlots.getMetadata}}\pysiglinewithargsret{\sphinxcode{pyleoclim.SummaryPlots.}\sphinxbfcode{getMetadata}}{\emph{timeseries}}{}
Get the necessary metadata to be printed out automatically
\begin{quote}\begin{description}
\item[{Parameters}] \leavevmode
\sphinxstyleliteralstrong{timeseries} -- a specific timeseries object.

\item[{Returns}] \leavevmode

archiveType

Authors (if more than 2, replace by et al.

PublicationYear

Publication DOI

Variable Name

Units

Climate Interpretation

Calibration Equation

Calibration References

Calibration Notes


\item[{Return type}] \leavevmode
A dictionary containing the following metadata

\end{description}\end{quote}

\end{fulllineitems}

\index{TsData() (in module pyleoclim.SummaryPlots)}

\begin{fulllineitems}
\phantomsection\label{\detokenize{SummaryPlots:pyleoclim.SummaryPlots.TsData}}\pysiglinewithargsret{\sphinxcode{pyleoclim.SummaryPlots.}\sphinxbfcode{TsData}}{\emph{timeseries}, \emph{x\_axis='`}}{}
Get the PaleoData with age/depth information

Get the necessary information for the TS plots/necessary to allow for
axes specification
\begin{quote}\begin{description}
\item[{Parameters}] \leavevmode\begin{itemize}
\item {} 
\sphinxstyleliteralstrong{timeseries} -- a single timeseries object.
By default, will prompt the user

\item {} 
\sphinxstyleliteralstrong{x-axis} (\sphinxstyleliteralemphasis{str}) -- The representation against which to plot the
paleo-data. Options are ``age'', ``year'', and ``depth''.
Default is to let the system choose if only one available
or prompt the user.

\end{itemize}

\item[{Returns}] \leavevmode

x - the x-valus

y - the y-values

archiveType - the archiveType (for plot settings)

x\_label - the label for the x-axis

y\_label - the label for the y-axis

label - the results of the x-axis query. Either depth, year, or age


\end{description}\end{quote}

\end{fulllineitems}

\index{agemodelData() (in module pyleoclim.SummaryPlots)}

\begin{fulllineitems}
\phantomsection\label{\detokenize{SummaryPlots:pyleoclim.SummaryPlots.agemodelData}}\pysiglinewithargsret{\sphinxcode{pyleoclim.SummaryPlots.}\sphinxbfcode{agemodelData}}{\emph{timeseries}}{}
Get the necessary information for the agemodel plot
\begin{quote}\begin{description}
\item[{Parameters}] \leavevmode
\sphinxstyleliteralstrong{timeseries} -- a single timeseries object. By default, will
prompt the user

\item[{Returns}] \leavevmode

depth - the depth values

age - the age values

x\_label - the label for the x-axis

y\_label - the label for the y-axis

archiveType - the archiveType (for default plot settings)


\end{description}\end{quote}

\end{fulllineitems}



\chapter{Spectral Functions}
\label{\detokenize{Spectral::doc}}\label{\detokenize{Spectral:spectral-functions}}
This module allows analysis in the frequency domain
\index{ar1\_fit() (in module pyleoclim.Spectral)}

\begin{fulllineitems}
\phantomsection\label{\detokenize{Spectral:pyleoclim.Spectral.ar1_fit}}\pysiglinewithargsret{\sphinxcode{pyleoclim.Spectral.}\sphinxbfcode{ar1\_fit}}{\emph{ys}, \emph{ts=None}, \emph{detrend='no'}}{}
Returns the lag-1 autocorrelation from ar1 fit OR persistence from tauest.
\begin{quote}\begin{description}
\item[{Parameters}] \leavevmode\begin{itemize}
\item {} 
\sphinxstyleliteralstrong{ys} (\sphinxstyleliteralemphasis{array}) -- the time series

\item {} 
\sphinxstyleliteralstrong{ts} (\sphinxstyleliteralemphasis{array}) -- the time axis of that series

\item {} 
\sphinxstyleliteralstrong{detrend} (\sphinxstyleliteralemphasis{str}) -- `no' - the original time series is assumed to have no trend;
`linear' - a linear least-squares fit to \sphinxtitleref{ys} is subtracted;
`constant' - the mean of \sphinxtitleref{ys} is subtracted

\end{itemize}

\item[{Returns}] \leavevmode
lag-1 autocorrelation coefficient (for evenly-spaced time series)
OR estimated persistence (for unevenly-spaced time series)

\item[{Return type}] \leavevmode
g (float)

\end{description}\end{quote}

\end{fulllineitems}

\index{ar1\_sim() (in module pyleoclim.Spectral)}

\begin{fulllineitems}
\phantomsection\label{\detokenize{Spectral:pyleoclim.Spectral.ar1_sim}}\pysiglinewithargsret{\sphinxcode{pyleoclim.Spectral.}\sphinxbfcode{ar1\_sim}}{\emph{ys}, \emph{n}, \emph{p}, \emph{ts=None}, \emph{detrend='no'}}{}
Produce p realizations of an AR1 process of length n with lag-1 autocorrelation g calculated from \sphinxtitleref{ys} and \sphinxtitleref{ts}
\begin{quote}\begin{description}
\item[{Parameters}] \leavevmode\begin{itemize}
\item {} 
\sphinxstyleliteralstrong{ys} (\sphinxstyleliteralemphasis{array}) -- a time series

\item {} 
\sphinxstyleliteralstrong{p} (\sphinxstyleliteralemphasis{n}\sphinxstyleliteralemphasis{,}\sphinxstyleliteralemphasis{}) -- dimensions as n rows by p columns

\item {} 
\sphinxstyleliteralstrong{ts} (\sphinxstyleliteralemphasis{array}) -- the time axis of that series

\item {} 
\sphinxstyleliteralstrong{detrend} (\sphinxstyleliteralemphasis{str}) -- `no' - the original time series is assumed to have no trend;
`linear' - a linear least-squares fit to \sphinxtitleref{ys} is subtracted;
`constant' - the mean of \sphinxtitleref{ys} is subtracted

\end{itemize}

\item[{Returns}] \leavevmode
n rows by p columns matrix of an AR1 process

\item[{Return type}] \leavevmode
red (matrix)

\end{description}\end{quote}

\end{fulllineitems}

\index{wwz() (in module pyleoclim.Spectral)}

\begin{fulllineitems}
\phantomsection\label{\detokenize{Spectral:pyleoclim.Spectral.wwz}}\pysiglinewithargsret{\sphinxcode{pyleoclim.Spectral.}\sphinxbfcode{wwz}}{\emph{ys}, \emph{ts}, \emph{tau=None}, \emph{freqs=None}, \emph{c=0.012665147955292222}, \emph{Neff=3}, \emph{nMC=200}, \emph{nproc=8}, \emph{detrend='no'}, \emph{method='Kirchner\_f2py'}}{}
Return the weighted wavelet amplitude (WWA) with phase, AR1\_q, and cone of influence, as well as WT coeeficients
\begin{quote}\begin{description}
\item[{Parameters}] \leavevmode\begin{itemize}
\item {} 
\sphinxstyleliteralstrong{ys} (\sphinxstyleliteralemphasis{array}) -- a time series, NaNs will be deleted automatically

\item {} 
\sphinxstyleliteralstrong{ts} (\sphinxstyleliteralemphasis{array}) -- the time points, if \sphinxtitleref{ys} contains any NaNs, some of the time points will be deleted accordingly

\item {} 
\sphinxstyleliteralstrong{tau} (\sphinxstyleliteralemphasis{array}) -- the evenly-spaced time points

\item {} 
\sphinxstyleliteralstrong{freqs} (\sphinxstyleliteralemphasis{array}) -- vector of frequency

\item {} 
\sphinxstyleliteralstrong{c} (\sphinxstyleliteralemphasis{float}) -- the decay constant, the default value 1/(8*np.pi**2) is good for most of the cases

\item {} 
\sphinxstyleliteralstrong{Neff} (\sphinxstyleliteralemphasis{int}) -- effective number of points

\item {} 
\sphinxstyleliteralstrong{nMC} (\sphinxstyleliteralemphasis{int}) -- the number of Monte-Carlo simulations

\item {} 
\sphinxstyleliteralstrong{nproc} (\sphinxstyleliteralemphasis{int}) -- the number of processes for multiprocessing

\item {} 
\sphinxstyleliteralstrong{detrend} (\sphinxstyleliteralemphasis{str}) -- `no' - the original time series is assumed to have no trend;
`linear' - a linear least-squares fit to \sphinxtitleref{ys} is subtracted;
`constant' - the mean of \sphinxtitleref{ys} is subtracted

\item {} 
\sphinxstyleliteralstrong{method} (\sphinxstyleliteralemphasis{str}) -- `Foster' - the original WWZ method;
`Kirchner' - the method Kirchner adapted from Foster;
`Kirchner\_f2py' - the method Kirchner adapted from Foster with f2py

\end{itemize}

\item[{Returns}] \leavevmode
the weighted wavelet amplitude.
AR1\_q (array): AR1 simulations
coi (array): cone of influence
freqs (array): vector of frequency
tau (array): the evenly-spaced time points, namely the time shift for wavelet analysis
Neffs (array): the matrix of effective number of points in the time-scale coordinates
coeff (array): the wavelet transform coefficents

\item[{Return type}] \leavevmode
wwa (array)

\end{description}\end{quote}

\end{fulllineitems}

\index{wwz\_psd() (in module pyleoclim.Spectral)}

\begin{fulllineitems}
\phantomsection\label{\detokenize{Spectral:pyleoclim.Spectral.wwz_psd}}\pysiglinewithargsret{\sphinxcode{pyleoclim.Spectral.}\sphinxbfcode{wwz\_psd}}{\emph{ys}, \emph{ts}, \emph{freqs=None}, \emph{tau=None}, \emph{c=0.001}, \emph{nproc=8}, \emph{nMC=200}, \emph{detrend='no'}, \emph{Neff=3}, \emph{anti\_alias=False}, \emph{avgs=2}, \emph{method='Kirchner\_f2py'}}{}
Return the psd of a timeseries directly using wwz method.
\begin{quote}\begin{description}
\item[{Parameters}] \leavevmode\begin{itemize}
\item {} 
\sphinxstyleliteralstrong{ys} (\sphinxstyleliteralemphasis{array}) -- a time series, NaNs will be deleted automatically

\item {} 
\sphinxstyleliteralstrong{ts} (\sphinxstyleliteralemphasis{array}) -- the time points, if \sphinxtitleref{ys} contains any NaNs, some of the time points will be deleted accordingly

\item {} 
\sphinxstyleliteralstrong{freqs} (\sphinxstyleliteralemphasis{array}) -- vector of frequency

\item {} 
\sphinxstyleliteralstrong{tau} (\sphinxstyleliteralemphasis{array}) -- the evenly-spaced time points, namely the time shift for wavelet analysis

\item {} 
\sphinxstyleliteralstrong{c} (\sphinxstyleliteralemphasis{float}) -- the decay constant, the default value 1e-3 is good for most of the cases

\item {} 
\sphinxstyleliteralstrong{nproc} (\sphinxstyleliteralemphasis{int}) -- the number of processes for multiprocessing

\item {} 
\sphinxstyleliteralstrong{nMC} (\sphinxstyleliteralemphasis{int}) -- the number of Monte-Carlo simulations

\item {} 
\sphinxstyleliteralstrong{detrend} (\sphinxstyleliteralemphasis{str}) -- `no' - the original time series is assumed to have no trend;
`linear' - a linear least-squares fit to \sphinxtitleref{ys} is subtracted;
`constant' - the mean of \sphinxtitleref{ys} is subtracted

\item {} 
\sphinxstyleliteralstrong{method} (\sphinxstyleliteralemphasis{str}) -- `Foster' - the original WWZ method;
`Kirchner' - the method Kirchner adapted from Foster;
`Kirchner\_f2py' - the method Kirchner adapted from Foster with f2py

\end{itemize}

\item[{Returns}] \leavevmode
power spectral density
freqs (array): vector of frequency
psd\_ar1\_q95 (array): the 95\% quantile of the psds of AR1 processes

\item[{Return type}] \leavevmode
psd (array)

\end{description}\end{quote}

\end{fulllineitems}

\index{plot\_wwa() (in module pyleoclim.Spectral)}

\begin{fulllineitems}
\phantomsection\label{\detokenize{Spectral:pyleoclim.Spectral.plot_wwa}}\pysiglinewithargsret{\sphinxcode{pyleoclim.Spectral.}\sphinxbfcode{plot\_wwa}}{\emph{wwa, freqs, tau, Neff=3, AR1\_q=None, coi=None, levels=None, tick\_range=None, yticks=None, ylim=None, xticks=None, xlabels=None, figsize={[}20, 8{]}, clr\_map='OrRd', cbar\_drawedges=False, cone\_alpha=0.5, plot\_signif=False, signif\_style='contour', plot\_cone=False, ax=None}}{}
Plot the wavelet amplitude
\begin{quote}\begin{description}
\item[{Parameters}] \leavevmode\begin{itemize}
\item {} 
\sphinxstyleliteralstrong{wwa} (\sphinxstyleliteralemphasis{array}) -- the weighted wavelet amplitude.

\item {} 
\sphinxstyleliteralstrong{freqs} (\sphinxstyleliteralemphasis{array}) -- vector of frequency

\item {} 
\sphinxstyleliteralstrong{tau} (\sphinxstyleliteralemphasis{array}) -- the evenly-spaced time points, namely the time shift for wavelet analysis

\item {} 
\sphinxstyleliteralstrong{Neff} (\sphinxstyleliteralemphasis{int}) -- effective number of points

\item {} 
\sphinxstyleliteralstrong{AR1\_q} (\sphinxstyleliteralemphasis{array}) -- AR1 simulations

\item {} 
\sphinxstyleliteralstrong{coi} (\sphinxstyleliteralemphasis{array}) -- cone of influence

\item {} 
\sphinxstyleliteralstrong{levels} (\sphinxstyleliteralemphasis{array}) -- levels of values to plot

\item {} 
\sphinxstyleliteralstrong{tick\_range} (\sphinxstyleliteralemphasis{array}) -- levels of ticks to show on the colorbar

\item {} 
\sphinxstyleliteralstrong{yticks} (\sphinxstyleliteralemphasis{list}) -- ticks on y-axis

\item {} 
\sphinxstyleliteralstrong{ylim} (\sphinxstyleliteralemphasis{list}) -- limitations for y-axis

\item {} 
\sphinxstyleliteralstrong{xticks} (\sphinxstyleliteralemphasis{list}) -- ticks on x-axis

\item {} 
\sphinxstyleliteralstrong{figsize} (\sphinxstyleliteralemphasis{list}) -- the size for the figure

\item {} 
\sphinxstyleliteralstrong{clr\_map} (\sphinxstyleliteralemphasis{str}) -- the name of the colormap

\item {} 
\sphinxstyleliteralstrong{cbar\_drawedges} (\sphinxstyleliteralemphasis{bool}) -- whether to draw edges on the colorbar or not

\item {} 
\sphinxstyleliteralstrong{cone\_alpha} (\sphinxstyleliteralemphasis{float}) -- the alpha value for the area covered by cone of influence

\item {} 
\sphinxstyleliteralstrong{plot\_signif} (\sphinxstyleliteralemphasis{bool}) -- plot 95\% significant area or not

\item {} 
\sphinxstyleliteralstrong{signif\_style} (\sphinxstyleliteralemphasis{str}) -- plot 95\% significant area with \sphinxtitleref{contour} or \sphinxtitleref{shade}

\item {} 
\sphinxstyleliteralstrong{plot\_cone} (\sphinxstyleliteralemphasis{bool}) -- plot cone of influence or not

\end{itemize}

\item[{Returns}] \leavevmode
the 2-D plot of wavelet analysis

\item[{Return type}] \leavevmode
fig (figure)

\end{description}\end{quote}

\end{fulllineitems}

\index{plot\_wwadist() (in module pyleoclim.Spectral)}

\begin{fulllineitems}
\phantomsection\label{\detokenize{Spectral:pyleoclim.Spectral.plot_wwadist}}\pysiglinewithargsret{\sphinxcode{pyleoclim.Spectral.}\sphinxbfcode{plot\_wwadist}}{\emph{wwa}}{}
Plot the distribution of wwa with the 95\% quantile line.
\begin{quote}\begin{description}
\item[{Parameters}] \leavevmode
\sphinxstyleliteralstrong{wwa} (\sphinxstyleliteralemphasis{array}) -- the weighted wavelet amplitude.

\item[{Returns}] \leavevmode
the 2-D plot of wavelet analysis

\item[{Return type}] \leavevmode
fig (figure)

\end{description}\end{quote}

\end{fulllineitems}

\index{plot\_psd() (in module pyleoclim.Spectral)}

\begin{fulllineitems}
\phantomsection\label{\detokenize{Spectral:pyleoclim.Spectral.plot_psd}}\pysiglinewithargsret{\sphinxcode{pyleoclim.Spectral.}\sphinxbfcode{plot\_psd}}{\emph{psd, freqs, lmstyle=None, linewidth=None, xticks=None, xlim=None, ylim=None, figsize={[}20, 8{]}, label='PSD', plot\_ar1=True, psd\_ar1\_q95=None, psd\_ar1\_color='\#d9544d', ax=None}}{}
Plot the wavelet amplitude
\begin{quote}\begin{description}
\item[{Parameters}] \leavevmode\begin{itemize}
\item {} 
\sphinxstyleliteralstrong{psd} (\sphinxstyleliteralemphasis{array}) -- power spectral density

\item {} 
\sphinxstyleliteralstrong{freqs} (\sphinxstyleliteralemphasis{array}) -- vector of frequency

\item {} 
\sphinxstyleliteralstrong{xticks} (\sphinxstyleliteralemphasis{list}) -- ticks on x-axis

\item {} 
\sphinxstyleliteralstrong{xlim} (\sphinxstyleliteralemphasis{list}) -- limitations for x-axis

\item {} 
\sphinxstyleliteralstrong{figsize} (\sphinxstyleliteralemphasis{list}) -- the size for the figure

\end{itemize}

\item[{Returns}] \leavevmode
the 2-D plot of wavelet analysis

\item[{Return type}] \leavevmode
fig (figure)

\end{description}\end{quote}

\end{fulllineitems}



\chapter{Indices and tables}
\label{\detokenize{index:indices-and-tables}}\begin{itemize}
\item {} 
\DUrole{xref,std,std-ref}{genindex}

\item {} 
\DUrole{xref,std,std-ref}{modindex}

\item {} 
\DUrole{xref,std,std-ref}{search}

\end{itemize}



\renewcommand{\indexname}{Index}
\printindex
\end{document}